\input{./src/main.sty}
% Additional SI unit for Fahrenheit
\DeclareSIUnit\fahrenheit{\degree F}
\sisetup{range-phrase = --}
\sisetup{range-units = single}

\begin{document}

% Include title page
\input{./src/titlepage.tex}

\begin{enumerate}
    \item Iron corrodes in a solution saturated with oxygen. The pH of the 
        solution is 5.5 and the partial pressure of Oxygen is 1.
    
        \begin{enumerate}[(i)]
            \item Calculate corrosion current and potential of the system.

            \item A sacrificial zinc-manganese alloy anode is coupled with
                iron to protect the structure.

                \begin{enumerate}[(a)]
                    \item Calculate the corrosion current of the protected
                        structure.
                    \item Calculate the corrosion current of the protected
                        structure
                \end{enumerate}

            \item Is the applied sacrificial protection system atisfactory?
                Does the system offer overprotection or underprotection?

        Given:

        \begin{align*}
            b_a &= \SI{0.1}{\volt\per decade}, &
            b_c &= \SI{-0.1}{\volt\per decade} \\
            P_{\ch{O2}} &= \SI{1}{atm} & \ch{[Fe^{2+}]} &= \SI{0.29}{M} \\
            e^o_{\ch{Fe}|\ch{Fe^{2+}}} &= \SI{-0.44}{\volt} \text{ vs. SHE} \\
            e^o_{\ch{ZnMn}} &= \SI{-1.46}{\volt} \text{ vs. SHE}
        \end{align*}

        Exchange current density for oxygen reduction
        $i^o_{\ch{O2}} = \SI{e-6}{\ampere\per\centi\meter\squared}$ \par
        Exchange current density for iron dissolution
        $i^o_{\ch{Fe}} = \SI{e-5}{\ampere\per\centi\meter\squared}$ \par
        Exchange current density for zinc alloy anode
        $i^o_{\ch{ZnMn}} = \SI{e-5}{\ampere\per\centi\meter\squared}$ \par

        \end{enumerate}
        
    \item Iron corrodes in a solution saturated with oxygen. The pH of the 
        solution plays a significant role in the corrosion of iron and the 
        applied current needed to protect the iron. To analyze the effect of 
        pH: 

        \begin{enumerate}[(i)]
            \item calculate the corrosion current and potential when the 
                solution has a pH of 14, 10, and 7;
            \item to bring down the corrosion current to the order of \num{e-5}
                , calculate the impressed current needed; and 
            \item Does the impressed current increase/decrease with pH? If so, 
                why?
        \end{enumerate}

        Given:

        \begin{align*}
            b_a &= \SI{0.1}{\volt\per decade}, & 
            b_c &= \SI{-0.1}{\volt\per decade} \\
            p_{\ch{O2}} &= \SI{1}{atm}, \ch{[Fe^{2+}]} &= \SI{0.5}{M}
        \end{align*}

        Exchange current density for oxygen reduction, 
        $i^o_{\ch{O2}} = \SI{e-7}{\ampere\per\centi\meter\squared}$ \par
        Exchange current density for iron dissolution, 
        $i^o_{\ch{Fe^{2+}}} = \SI{e-5}{\ampere\per\centi\meter\squared}$ \par

        \textbf{Solution:}\par
        Anodic reaction: \ch{Fe -> Fe^{2+} + 2 e^-}, $e^o = \SI{0.44}{\volt}$

\end{enumerate}

\end{document}

% Additional SI unit for Fahrenheit
\DeclareSIUnit\fahrenheit{\degree F}
\sisetup{range-phrase = --}
\sisetup{range-units = single}

\begin{document}

% Include title page
\input{./src/titlepage.tex}

\begin{enumerate}
    \item Iron corrodes in a solution saturated with oxygen. The pH of the 
        solution is 5.5 and the partial pressure of Oxygen is 1.
    
        \begin{enumerate}[(i)]
            \item Calculate corrosion current and potential of the system.

            \item A sacrificial zinc-manganese alloy anode is coupled with
                iron to protect the structure.

                \begin{enumerate}[(a)]
                    \item Calculate the corrosion current of the protected
                        structure.
                    \item Calculate the corrosion current of the protected
                        structure
                \end{enumerate}

            \item Is the applied sacrificial protection system atisfactory?
                Does the system offer overprotection or underprotection?

        Given:

        \begin{align*}
            b_a &= \SI{0.1}{\volt\per decade}, &
            b_c &= \SI{-0.1}{\volt\per decade} \\
            P_{\ch{O2}} &= \SI{1}{atm} & \ch{[Fe^{2+}]} &= \SI{0.29}{M} \\
            e^o_{\ch{Fe}|\ch{Fe^{2+}}} &= \SI{-0.44}{\volt} \text{ vs. SHE} \\
            e^o_{\ch{ZnMn}} &= \SI{-1.46}{\volt} \text{ vs. SHE}
        \end{align*}

        Exchange current density for oxygen reduction
        $i^o_{\ch{O2}} = \SI{e-6}{\ampere\per\centi\meter\squared}$ \par
        Exchange current density for iron dissolution
        $i^o_{\ch{Fe}} = \SI{e-5}{\ampere\per\centi\meter\squared}$ \par
        Exchange current density for zinc alloy anode
        $i^o_{\ch{ZnMn}} = \SI{e-5}{\ampere\per\centi\meter\squared}$ \par

        \end{enumerate}
        
    \item Iron corrodes in a solution saturated with oxygen. The pH of the 
        solution plays a significant role in the corrosion of iron and the 
        applied current needed to protect the iron. To analyze the effect of 
        pH: 

        \begin{enumerate}[(i)]
            \item calculate the corrosion current and potential when the 
                solution has a pH of 14, 10, and 7;
            \item to bring down the corrosion current to the order of \num{e-5}
                , calculate the impressed current needed; and 
            \item Does the impressed current increase/decrease with pH? If so, 
                why?
        \end{enumerate}

        Given:

        \begin{align*}
            b_a &= \SI{0.1}{\volt\per decade}, & 
            b_c &= \SI{-0.1}{\volt\per decade} \\
            p_{\ch{O2}} &= \SI{1}{atm}, \ch{[Fe^{2+}]} &= \SI{0.5}{M}
        \end{align*}

        Exchange current density for oxygen reduction, 
        $i^o_{\ch{O2}} = \SI{e-7}{\ampere\per\centi\meter\squared}$ \par
        Exchange current density for iron dissolution, 
        $i^o_{\ch{Fe^{2+}}} = \SI{e-5}{\ampere\per\centi\meter\squared}$ \par

        \textbf{Solution:}\par
        Anodic reaction: \ch{Fe -> Fe^{2+} + 2 e^-}, $e^o = \SI{0.44}{\volt}$

\end{enumerate}

\end{document}

% Additional SI unit for Fahrenheit
\DeclareSIUnit\fahrenheit{\degree F}
\sisetup{range-phrase = --}
\sisetup{range-units = single}

\begin{document}

% Include title page
\input{./src/titlepage.tex}

\begin{enumerate}
    \item Iron corrodes in a solution saturated with oxygen. The pH of the 
        solution is 5.5 and the partial pressure of Oxygen is 1.
    
        \begin{enumerate}[(i)]
            \item Calculate corrosion current and potential of the system.

            \item A sacrificial zinc-manganese alloy anode is coupled with
                iron to protect the structure.

                \begin{enumerate}[(a)]
                    \item Calculate the corrosion current of the protected
                        structure.
                    \item Calculate the corrosion current of the protected
                        structure
                \end{enumerate}

            \item Is the applied sacrificial protection system atisfactory?
                Does the system offer overprotection or underprotection?

        Given:

        \begin{align*}
            b_a &= \SI{0.1}{\volt\per decade}, &
            b_c &= \SI{-0.1}{\volt\per decade} \\
            P_{\ch{O2}} &= \SI{1}{atm} & \ch{[Fe^{2+}]} &= \SI{0.29}{M} \\
            e^o_{\ch{Fe}|\ch{Fe^{2+}}} &= \SI{-0.44}{\volt} \text{ vs. SHE} \\
            e^o_{\ch{ZnMn}} &= \SI{-1.46}{\volt} \text{ vs. SHE}
        \end{align*}

        Exchange current density for oxygen reduction
        $i^o_{\ch{O2}} = \SI{e-6}{\ampere\per\centi\meter\squared}$ \par
        Exchange current density for iron dissolution
        $i^o_{\ch{Fe}} = \SI{e-5}{\ampere\per\centi\meter\squared}$ \par
        Exchange current density for zinc alloy anode
        $i^o_{\ch{ZnMn}} = \SI{e-5}{\ampere\per\centi\meter\squared}$ \par

        \end{enumerate}
        
    \item Iron corrodes in a solution saturated with oxygen. The pH of the 
        solution plays a significant role in the corrosion of iron and the 
        applied current needed to protect the iron. To analyze the effect of 
        pH: 

        \begin{enumerate}[(i)]
            \item calculate the corrosion current and potential when the 
                solution has a pH of 14, 10, and 7;
            \item to bring down the corrosion current to the order of \num{e-5}
                , calculate the impressed current needed; and 
            \item Does the impressed current increase/decrease with pH? If so, 
                why?
        \end{enumerate}

        Given:

        \begin{align*}
            b_a &= \SI{0.1}{\volt\per decade}, & 
            b_c &= \SI{-0.1}{\volt\per decade} \\
            p_{\ch{O2}} &= \SI{1}{atm}, \ch{[Fe^{2+}]} &= \SI{0.5}{M}
        \end{align*}

        Exchange current density for oxygen reduction, 
        $i^o_{\ch{O2}} = \SI{e-7}{\ampere\per\centi\meter\squared}$ \par
        Exchange current density for iron dissolution, 
        $i^o_{\ch{Fe^{2+}}} = \SI{e-5}{\ampere\per\centi\meter\squared}$ \par

        \textbf{Solution:}\par
        Anodic reaction: \ch{Fe -> Fe^{2+} + 2 e^-}, $e^o = \SI{0.44}{\volt}$

\end{enumerate}

\end{document}

% Additional SI unit for Fahrenheit
\DeclareSIUnit\fahrenheit{\degree F}
\sisetup{range-phrase = --}
\sisetup{range-units = single}

\begin{document}

% Include title page
\input{./src/titlepage.tex}

\begin{enumerate}
    \item Iron corrodes in a solution saturated with oxygen. The pH of the 
        solution is 5.5 and the partial pressure of Oxygen is 1.
    
        \begin{enumerate}[(i)]
            \item Calculate corrosion current and potential of the system.

            \item A sacrificial zinc-manganese alloy anode is coupled with
                iron to protect the structure.

                \begin{enumerate}[(a)]
                    \item Calculate the corrosion current of the protected
                        structure.
                    \item Calculate the corrosion current of the protected
                        structure
                \end{enumerate}

            \item Is the applied sacrificial protection system atisfactory?
                Does the system offer overprotection or underprotection?

        Given:

        \begin{align*}
            b_a &= \SI{0.1}{\volt\per decade}, &
            b_c &= \SI{-0.1}{\volt\per decade} \\
            P_{\ch{O2}} &= \SI{1}{atm} & \ch{[Fe^{2+}]} &= \SI{0.29}{M} \\
            e^o_{\ch{Fe}|\ch{Fe^{2+}}} &= \SI{-0.44}{\volt} \text{ vs. SHE} \\
            e^o_{\ch{ZnMn}} &= \SI{-1.46}{\volt} \text{ vs. SHE}
        \end{align*}

        Exchange current density for oxygen reduction
        $i^o_{\ch{O2}} = \SI{e-6}{\ampere\per\centi\meter\squared}$ \par
        Exchange current density for iron dissolution
        $i^o_{\ch{Fe}} = \SI{e-5}{\ampere\per\centi\meter\squared}$ \par
        Exchange current density for zinc alloy anode
        $i^o_{\ch{ZnMn}} = \SI{e-5}{\ampere\per\centi\meter\squared}$ \par

        \end{enumerate}
        
    \item Iron corrodes in a solution saturated with oxygen. The pH of the 
        solution plays a significant role in the corrosion of iron and the 
        applied current needed to protect the iron. To analyze the effect of 
        pH: 

        \begin{enumerate}[(i)]
            \item calculate the corrosion current and potential when the 
                solution has a pH of 14, 10, and 7;
            \item to bring down the corrosion current to the order of \num{e-5}
                , calculate the impressed current needed; and 
            \item Does the impressed current increase/decrease with pH? If so, 
                why?
        \end{enumerate}

        Given:

        \begin{align*}
            b_a &= \SI{0.1}{\volt\per decade}, & 
            b_c &= \SI{-0.1}{\volt\per decade} \\
            p_{\ch{O2}} &= \SI{1}{atm}, \ch{[Fe^{2+}]} &= \SI{0.5}{M}
        \end{align*}

        Exchange current density for oxygen reduction, 
        $i^o_{\ch{O2}} = \SI{e-7}{\ampere\per\centi\meter\squared}$ \par
        Exchange current density for iron dissolution, 
        $i^o_{\ch{Fe^{2+}}} = \SI{e-5}{\ampere\per\centi\meter\squared}$ \par

        \textbf{Solution:}\par
        Anodic reaction: \ch{Fe -> Fe^{2+} + 2 e^-}, $e^o = \SI{0.44}{\volt}$

\end{enumerate}

\end{document}
